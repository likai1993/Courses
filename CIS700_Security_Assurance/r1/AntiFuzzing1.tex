\documentclass[letterpaper,11pt]{article}
\usepackage{tabularx} % extra features for tabular environment
%\usepackage{amsmath}  % improve math presentation
%\usepackage{graphicx} % takes care of graphic including machinery
\usepackage[margin=1in,letterpaper]{geometry} % decreases margins
%\usepackage{cite} % takes care of citations
\usepackage[final]{hyperref} % adds hyper links inside the generated pdf file
\hypersetup{
	colorlinks=true,       % false: boxed links; true: colored links
	linkcolor=blue,        % color of internal links
	citecolor=blue,        % color of links to bibliography
	filecolor=magenta,     % color of file links
	urlcolor=blue         
}
\usepackage{blindtext}
%++++++++++++++++++++++++++++++++++++++++

\begin{document}

\begin{center}
{\Large FuzziFication: Anti-Fuzzing Techniques} 

{Jinho Jung, Hong Hu, David Solodukhin, Daniel Pagan, Kyu Hyung Lee, and Taesoo Kim}
\bigskip

{\large Writer: Kai Li}
\date{\today}
\end{center}


\section{Summary}
This paper proposed an anti-fuzzing system \textbf{Fuzzification}~\cite{reading1} which contains three anti-fuzzing techniques that can slow down the state-of-art fuzzers and reduce the coverage by the fuzzers. Developers can use these techniques to protect their programs when exposing it to the public where attackers are equipped with state-of-art fuzzers. Slow down the attackers can benefit the developers as they can find bugs more quickly. 

The three techniques are so-called \textbf{SpeedBump}, \textbf{BranchTrap}, and \textbf{AntiHybrid}. 
\begin{itemize}
\item SpeedBump works by adding configurable overhead to those cold paths using CSmith by generating delayed blocks that have control-flow dependency and data-flow dependency to the original code blocks, thus it is difficult to be detected by the fuzzers.    
\item BranchTrap works by chaining various small code snippets and reusing the return-oriented programming intuition. This technique can generate branches that sensitive to the input, thus fool the fuzzers to go the wrong directions which would cover those faked code paths.
\item AntiHybrid works by introducing implicit data-flow dependencies and injecting multiple code chunks to intentionally trigger path explosions. This technique can hinder fuzzers equipped with symbolic-execution (or dynamic taint analysis). 
\end{itemize}

Evaluation results show that compared with existing fuzzing techniques, \textbf{Fuzzification} can reduce the coverage by 76\% (to AFL) and 67\% (to HonggFuzz) on average, reduce the discovered crashed by 88\% (to AFL), 98\% (to HonggFuzz), 94\% (to QSym), while incurs negligible overhead (5.4\% in code size and 0.73\% in run time) to the normal users.     

\section{Strengths and Weaknesses}
\subsection {Strengths}
1. The proposed system light a new research direction on anti-fuzzing, which can be further explored.\\
2. Three novel techniques are proposed to anti-fuzzing, e.g., slow down fuzzing execution, hide path coverage and hinder dynamic taint-analysis and symbolic execution.\\
3. Strong evaluation result supports the authors' claim that the techniques work effectively to slow down the exising fuzzers and reduce the coverage, which introduces small overhead to the normal users. 

\subsection {Weakness}
1. A major limitation of this work is the defensive methods, if the attacker is knowledgable with these anti-fuzzing techniques, they can easily bypass the protections, and these anti-fuzzing techniques need to be improved continuously along the back-and-forth attack and defense progress.\\ 
2. These techniques standalone cannot effectively eliminate coverage by the existing fuzzers, especially for fuzzers running with powerful computing resources, these anti-fuzzing techniques need to be combined with other mitigations.\\ 
3. The evaluation result on small programs (LAVA-M programs) shows \textbf{Fuzzification} brings significant overhead and not effective to small programs.

\bibliographystyle{plain}
\bibliography{readings}  
\end{document}
